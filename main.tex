\documentclass[12pt,a4paper]{article}
\usepackage[UTF8,fontset=none]{ctex}
\usepackage{xeCJK}
\usepackage{fontspec}
\setCJKmainfont{Noto Sans CJK TC}
\setCJKsansfont{Noto Sans CJK TC}
\setCJKmonofont{Noto Sans CJK TC}
\setmainfont{Times New Roman}
\setsansfont{Times New Roman}
\setmonofont{Courier New}
\usepackage{amsmath}
\usepackage{graphicx}
\usepackage{geometry}
\usepackage{listings}
\usepackage{xcolor}
\usepackage{hyperref}
\usepackage{float}

\geometry{margin=2.5cm}

\lstset{
    language=Python,
    basicstyle=\ttfamily\small,
    keywordstyle=\color{blue},
    commentstyle=\color{green!60!black},
    stringstyle=\color{red},
    numbers=left,
    numberstyle=\tiny\color{gray},
    stepnumber=1,
    numbersep=5pt,
    frame=single,
    breaklines=true,
    showstringspaces=false
}

\title{計算機概論 Final Project\\泰勒級數逼近之數值分析與視覺化}
\author{}
\date{}

\begin{document}

\renewcommand{\figurename}{圖}
\renewcommand{\tablename}{表}

\maketitle

\vspace{1.5cm}
\begin{center}
\begin{tabular}{ll}
    \textbf{姓名:} & 周明坤 \\[0.3cm]
    \textbf{科系:} & 不分系 \\[0.3cm]
    \textbf{學號:} & AN4096750 \\[0.3cm]
    \textbf{GitHub:} & \url{https://github.com/MKChou/Taylor-Series-Aproximation}
\end{tabular}
\end{center}
\vspace{1.5cm}

\newpage
\tableofcontents
\newpage

\section{問題定義與理解 (Problem Definition \& Understanding)}

電腦其實沒辦法直接計算 $\sin(x)$、$\cos(x)$ 或 $e^x$ 這些複雜的函數。所以我們需要用\textbf{多項式近似(Polynomial Approximation)}的方法來計算出這些函數的值。

\textbf{泰勒級數(Taylor Series)}就是把一個函數拆解成很多項多項式加起來。這次專案我選了 $\sin(x)$ 來研究,用 Python 寫程式看看不同項數的泰勒級數能多準確地逼近真實的 $\sin(x)$,還有誤差會怎麼變化。

\subsection{為什麼電腦需要逼近法?}

\begin{enumerate}
    \item \textbf{硬體限制:} 電腦只會做加減乘除,沒辦法直接計算 $\sin(x)$ 這種函數。
    \item \textbf{效率考量:} 如果每個函數都要用查表或硬體來做,成本太高了。用逼近法可以根據需要的精度來決定要算幾項,使用起來比較彈性。
    \item \textbf{實用性:} 透過調整項數可以在準確度和計算時間之間找到平衡。
\end{enumerate}

\section{數學背景 (Mathematical Model)}

$\sin(x)$ 在 $x = 0$ 處的麥克勞林級數(Maclaurin Series,泰勒級數的一種特例)公式如下:

\begin{equation}
\sin(x) = \sum_{i=0}^{\infty} \frac{(-1)^i}{(2i+1)!} x^{2i+1}
\end{equation}

展開後為:

\begin{equation}
\sin(x) = x - \frac{x^3}{3!} + \frac{x^5}{5!} - \frac{x^7}{7!} + \frac{x^9}{9!} - \cdots
\end{equation}

實際上我們不可能計算無限項,所以只能取前 $n$ 項來近似。這次專案我試了不同的 $n$ 值,看看效果如何。

\textbf{通項公式:}
\begin{equation}
T_i = \frac{(-1)^i}{(2i+1)!} x^{2i+1}
\end{equation}

其中:
\begin{itemize}
    \item $i$ 為項數 index(從 0 開始)
    \item $(2i+1)!$ 為階乘
    \item $(-1)^i$ 提供交替的正負號
\end{itemize}

\subsection{收斂性分析}

\begin{itemize}
    \item \textbf{收斂半徑:} $\sin(x)$ 的麥克勞林級數對所有實數 $x$ 都收斂(收斂半徑為無窮大)。
    \item \textbf{誤差特性:} 當 $|x|$ 較小時,級數收斂較快;當 $|x|$ 較大時,需要更多項才能達到相同精度。
\end{itemize}

\section{程式實作與效率 (Programming \& Efficiency)}

這次專案我用 Python 來寫,主要用了幾個工具:

\begin{enumerate}
    \item \textbf{NumPy 的向量化運算:} 不用 Python 的 \texttt{for} 迴圈一個一個算,而是用 NumPy 陣列一次處理整個範圍的 $x$ 值。我設定了 $x$ 從 $-2\pi$ 到 $2\pi$,取 1000 個點,用向量化運算可以一次處理所有點,這樣會快很多。
    
    \item \textbf{SciPy 的階乘函數:} 用 \texttt{scipy.special.factorial} 來算階乘,這樣即使項數很多也不會出錯。例如當 $n=8$ 時,需要計算到 $15!$,這個數字很大,用 SciPy 的函數可以確保數值穩定。
    
    \item \textbf{Matplotlib 畫圖:} 用 Matplotlib 把結果畫成圖表,比較容易看出來。我畫了兩個子圖,一個是函數比較圖,一個是誤差分析圖,都用對數尺度來顯示,這樣可以同時看到小誤差和大誤差的變化。
\end{enumerate}

\subsection{實驗設置}

這次實驗我設定了以下參數:
\begin{itemize}
    \item $x$ 的範圍:從 $-2\pi$ 到 $2\pi$,共 1000 個點
    \item 測試的項數:$n = 1, 3, 5, 8$
    \item 誤差閾值:0.01(用於進階分析)
\end{itemize}

\subsection{為什麼使用向量化運算?}

\textbf{向量化運算}是 NumPy 的重點功能,好處是:

\begin{enumerate}
    \item \textbf{算得比較快:}
    \begin{itemize}
        \item Python 的迴圈很慢,向量化運算可以一次處理很多資料
        \item NumPy 底層是用 C 寫的,所以很快
        \item 可以一次對很多數字做同樣的運算
    \end{itemize}
    
    \item \textbf{程式碼比較簡單:} 不用寫迴圈,程式碼看起來比較清楚。
    
    \item \textbf{比較省記憶體:} 不需要建立很多中間變數。
\end{enumerate}

\section{數據分析與視覺化 (Data Analysis \& Visualization)}

\subsection{逼近曲線分析}

從圖表(圖~\ref{fig:approximation} 左側)可以看出:

\begin{itemize}
    \item \textbf{n=1(線性逼近):} 這是 $y = x$ 的直線,只有在 $x$ 很小的時候才準,離 0 遠一點就開始偏了。當 $|x| > 0.5$ 時,誤差就明顯可見。
    \item \textbf{n=3(三次逼近):} 加入了 $x^3$ 項,曲線開始有彎曲的形狀,範圍變大一點,大概到 $|x| < \pi$ 都還算準。但超過這個範圍後,誤差會快速增加。
    \item \textbf{n=5(五次逼近):} 逼近效果更好,在 $|x| < 2\pi$ 的範圍內都能很好地逼近 $\sin(x)$。
    \item \textbf{n=8(八次逼近):} 範圍更大,跟真正的 $\sin(x)$ 幾乎重疊,在整個 $[-2\pi, 2\pi]$ 範圍內都很準確。
    \item \textbf{結論:} 項數越多,能準確逼近的範圍就越大。從圖中可以清楚看到,隨著項數增加,逼近曲線越來越接近真實的 $\sin(x)$ 曲線。
\end{itemize}

\subsection{誤差行為分析}

為了看清楚誤差有多大,我畫了\textbf{絕對誤差圖},用對數尺度來看(圖~\ref{fig:approximation} 右側):

\begin{itemize}
    \item \textbf{靠近 0 的地方:} 誤差很小,大概 $10^{-15}$ 左右,幾乎是電腦能算到最準的程度了。這表示在 $x=0$ 附近,泰勒級數的逼近效果非常好。
    \item \textbf{離 0 越遠:} 誤差就越大,而且長得很快。從圖中可以看到誤差呈現指數級成長的趨勢。
    \item \textbf{項數的取捨:} 項數多一點誤差會比較小,但算的時間也會變長,這就是準確度和速度之間的取捨。
    \item \textbf{誤差閾值分析:} 根據程式的進階分析,當誤差閾值設為 0.01 時,n=1 項只能在 $|x| < 0.38$ 的範圍內達到要求;n=3 項可以擴展到 $|x| < 1.75$;n=5 項可以到 $|x| < 3.24$;n=8 項則可以達到 $|x| < 5.49$。這清楚地展示了項數增加對誤差控制的影響。
\end{itemize}

\begin{figure}[H]
    \centering
    \includegraphics[width=0.9\textwidth]{result.png}
    \caption{sin(x) 泰勒級數逼近比較與誤差分析圖}
    \label{fig:approximation}
\end{figure}

\section{結論 (Conclusion)}

這次專案讓我用 Python 把微積分的理論實際跑了一遍,從中學到了很多。

\begin{enumerate}
    \item \textbf{理論驗證:} 結果跟學的泰勒級數一樣,證明用多項式來逼近是可行的。實驗結果清楚地展示了泰勒級數的收斂特性,項數越多逼近效果越好。
    
    \item \textbf{工具學習:} 學會用 NumPy 和 SciPy,發現它們比直接用 Python 的列表快很多。向量化運算讓程式碼更簡潔,執行效率也大幅提升。
    
    \item \textbf{實用經驗:} 知道在寫程式時,要根據需要的準確度和計算時間來決定要用幾項,這是這次學到的重要經驗。不同的應用場景需要不同的精度,不能一味追求高精度而忽略計算成本。
    
    \item \textbf{視覺化的重要性:} 透過圖表可以很直觀地看到誤差的變化趨勢,這比單純看數字更容易理解。對數尺度的使用讓小誤差和大誤差都能清楚呈現。
\end{enumerate}

\subsection{精確度與計算成本的平衡}

從實驗結果可以得出幾個結論:

\begin{enumerate}
    \item \textbf{項數和準確度的關係:}
    \begin{itemize}
        \item 項數越多算得越準,能用的範圍也越大
        \item 但是項數多就要算比較久
    \end{itemize}
    
    \item \textbf{$x$ 的範圍影響:}
    \begin{itemize}
        \item 如果 $x$ 很小(靠近 0),用幾項就很準了
        \item 如果 $x$ 比較大,就要用更多項才能維持準確度
    \end{itemize}
    
    \item \textbf{實際應用建議:}
    \begin{itemize}
        \item 如果 $|x| < 1$:用 3-5 項就夠了,誤差可以控制在很小的範圍內。
        \item 如果 $|x| < 2\pi$:建議用 8-10 項,這樣可以保證在整個範圍內都有良好的精度。
        \item 如果範圍更大:需要更多項,或者考慮其他方法,比如先將 $x$ 縮小到 $[-\pi, \pi]$ 範圍內再計算。
    \end{itemize}
    
    \item \textbf{未來改進方向:}
    \begin{itemize}
        \item 可以嘗試使用其他數值方法,如 CORDIC 演算法,在某些情況下可能更有效率。
        \item 可以實作自適應項數選擇,根據輸入的 $x$ 值自動決定需要多少項。
        \item 可以比較不同逼近方法的計算效率和精度。
    \end{itemize}
\end{enumerate}

\end{document}

